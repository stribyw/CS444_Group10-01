\documentclass[letterpaper,10pt,serif,draftclsnofoot,onecolumn,compsoc,titlepage]{IEEEtran}

\usepackage{graphicx}
\usepackage{amssymb}
\usepackage{amsmath}
\usepackage{amsthm}
\usepackage{cite}
\usepackage{alltt}
\usepackage{float}
\usepackage{color}
\usepackage{url}

\usepackage{balance}
\usepackage[TABBOTCAP, tight]{subfigure}
\usepackage{enumitem}

\usepackage{geometry}
\geometry{margin=.75in}
\usepackage{hyperref}
%\usetikzlibrary{shapes, positioning, calc}
\usepackage{caption}
\usepackage{listings}
%\usepackage[utf8]{inputenc}
%pull in the necessary preamble matter for pygments output

%% The following metadata will show up in the PDF properties
% \hypersetup{
%   colorlinks = false,
%   urlcolor = black,
%   pdfauthor = {\name},
%   pdfkeywords = {cs311 ``operating systems'' files filesystem I/O},
%   pdftitle = {CS 311 Project 1: UNIX File I/O},
%   pdfsubject = {CS 311 Project 1},
%   pdfpagemode = UseNone
% }

\parindent = 0.0 in
\parskip = 0.1 in
\title{Writing Assignment 1}
\author{Shannon Ernst, Keith Stirby, Tanner Cecchetti\\ 5 May 2017 \\ CS 444 Spring 2017 \\ Group 10-01}
\begin{document}
\maketitle
\begin{abstract}

This document explores the design of our C-LOOK scheduler implementation and
outlines our testing procedures. Additionally, it includes a work log and
version control log which track our progress.

\end{abstract}
\newpage
\tableofcontents
\newpage
\section{Purpose and Learning}
The purpose of this assignment was to explore patching the kernel with a
new I/O scheduler, in particular the LOOK scheduler. The LOOK scheduler is
an elevator algorithm which is based on traveling in one direction on the
disk, looking to see if there is anything left in that direction, if there
is not, it turns around and continues going in the opposite direction
until nothing is left and so on. C-Look is similar only it goes to the
top and then restarts at the bottom instead of traveling down, servicing
requests along the way. From this assignment we learned a number of things.
We learned primarily about search the kernel code base for things that
were relevant to our patch. This was important as often the hardest step
is not figuring out the solution to the problem (we all had a fairly good
idea of what the elevator should do) but how to do it within the code
constraints presented. We were instructed to use the noop io scheduler file
as our guide so the first thing we had to do was figure out what was going
on in the file and what to change. This was the longest part of the
assignment. The next thing we learned was how to test that the system is
working. Not surprisingly, it is fairly difficult to test things which
are going on in the background or happening rapidly. Last we learned about
the elevator algorithm and queues in the kernal. Though this was the main
thrust of the assignment, it is ranked last as we definitely spent more
time and learned more on the supporting structures.

\section{Design Choices}
In designing our solution we had a couple decisions to make. The first was
whether to implement LOOK or C-LOOK. The difference between the two
approaches is LOOK requires that the list is sorted relative to the current
position of the head on insert where C-LOOK is simply sorting in ascending
order. From an efficiency standpoint, LOOK is the better option. What
would need to occur in LOOK is we would have to pick a turning point for
going from ascending to descending and vice versa. This could be
accomplished by how we sort and insert. If a value came in that was less
than our head it would be placed after our max value but sorted relative to
all of the values after our max. If a value came in that was greater than
our head but less than our max, then it would be sorted between the head
and the max. If a value came in that less than our smallest number it would
appended after our smallest number and point to the next highest if
we were in the ascending portion of the queue. This constant switching
and monitoring of direction can be tricky to do and will require two kinds
of sorts. \\
For C-LOOK, it is slightly less efficient because it will go in one
direction until it runs out of jobs and then restarts from the smallest
number. This means the disk head will make a full pass of the disk without
doing anything. The algorithm for implementing this a tad simpler with only
one sort, a classic insertion sort, with the max number pointing next to
the smallest number. The only special cases for the sort would be if
inserting a number greater than the max or smaller than the smallest which
would just be getting the smallest previous to point to the max and the
max next to point to the smallest. In either case of LOOK or C-LOOK the
dispatch stays the same, only having to follow the next pointers of the
queue as everything is in the correct order. \\
We decided to go with the C-LOOK as it was an easier algorithm for us to
understand from a coding perspective, despite it being slightly less
efficient. The next design choice we had to make was how we were going
to keep track of our queue. We used the kernel data structure for this and
some kernel code to get access to the sector. Ultimately, we did not have
to change much in the noop io scheduler aside from add request which simply
required us to specify our method of inserting as we described above.
\section{Testing}
The first indication that everything was working was the kernel boot. For
the first few times the kernel would not compile or boot. The next thing
we did was print the sector number being added and dispatched. This would
just print to the screen and interrupt everything in the foreground.
This gave us a good indication that things were happening in order but
was generally hard to read. We finally settled on creating a python
script which generated 25 files and one million characters in
each. We plotted our output and saw that it was working correctly but were
concerned that our testing wasn't rigorous enough. We decided to redo our
script so that we created the files, shuffled their order, and randomly
read and wrote inside the file via seek. This did confirm that we were
testing correctly. Our graph indicates that we do jump all over the disk
but only in regards to files not the actual sectors.
\section{Work Log}
Tuesday, May 2, 3:20 pm to 5:30 pm: We met as a group to begin the
assignment. The majority of the time was spent researching and disecting
the kernel code. We did not know where to begin. We did the initial set up
of copying the noop file and and doing the proper adjustments of sstf but
that was about it. \\
Wednesday, May 3, 6:00 pm to 9:00 pm: We met to continue working on
the assignment. We began by grabbing a whiteboard and breaking down the
LOOK and C-LOOK algorithm and making a decision to go with C-LOOK. We
then spent time wrapping our head around how queues work in the kernel
and how to do a sort that would reflect C-LOOK and LOOK. Once we had
determined this we began coding. We had figured out where we needed to
adjust the code (in add request). We tested a few times and things looked
to be working so we called it a night. \\
Thursday, May 4, 6:00 pm to 10:00 pm: We met to continue working. We were
in the testing stage. We wrote some python scripts to generate file i/o.
We also found a way to plot our output with python so we could see it
graphically. We began to worry that we were not doing things correctly
because the graph looked to clean. We adjusted our scripts and caught
one bug that we fixed. We were still concerned so we emailed McGrath our
data and he said it was correct.\\
Friday, May 5, 2:00 pm to 3:30 pm: We met after our demo to finish the
code. The concurrency had been completed during recitation two and we
 checked it to make sure it was working. We implemented our patch
  and submitted. \\

\section{Version Control Log}

\begin{tabular}{ | p{1.5cm} | p{4cm} | p{3cm} | p{8cm} | }\hline\textbf{Commit} & \textbf{Date} & \textbf{Author} & \textbf{Description}\\\hline
\href{https://github.com/stribyw/CS444_producer-consumer/commit/ea7fef3257e2c4c38b2c29083c0b9e1ac99e4e97}{ea7fef3} & \detokenize{05 May 2017 03:16 PM} & \detokenize{Tanner Cecchetti} & \detokenize{Add patch file}\\\hline
\href{https://github.com/stribyw/CS444_producer-consumer/commit/1420d4c1a7a833c8843e4b8666be886535355648}{1420d4c} & \detokenize{05 May 2017 03:08 PM} & \detokenize{William StribyJr} & \detokenize{remove files}\\\hline
\href{https://github.com/stribyw/CS444_producer-consumer/commit/b1468d2d9a73e42ae338de81b257cc980bbfeb19}{b1468d2} & \detokenize{05 May 2017 03:05 PM} & \detokenize{William StribyJr} & \detokenize{altered test_random_3 and renamed the tests}\\\hline
\href{https://github.com/stribyw/CS444_producer-consumer/commit/15427d1ae5fd4e2cf5b51f07cadeb219ddc01c08}{15427d1} & \detokenize{05 May 2017 02:37 PM} & \detokenize{William StribyJr} & \detokenize{update local for push Merge branch 'master' of https://github.com/stribyw/CS444_Group10-01}\\\hline
\href{https://github.com/stribyw/CS444_producer-consumer/commit/69e1462acee8e054d55f1a9b3a9f859bf5626bc2}{69e1462} & \detokenize{05 May 2017 02:37 PM} & \detokenize{William StribyJr} & \detokenize{added more test scripts}\\\hline
\href{https://github.com/stribyw/CS444_producer-consumer/commit/685bd8aab81cdd9e665269f9950d8a634683d007}{685bd8a} & \detokenize{05 May 2017 02:11 PM} & \detokenize{Shannon Ernst} & \detokenize{Merge branch 'master' of https://github.com/stribyw/CS444_Group10-01}\\\hline
\href{https://github.com/stribyw/CS444_producer-consumer/commit/48c6c968122cb151d1e8c67df796998c8ea771bd}{48c6c96} & \detokenize{05 May 2017 02:10 PM} & \detokenize{Shannon Ernst} & \detokenize{finishing write up}\\\hline
\href{https://github.com/stribyw/CS444_producer-consumer/commit/6ac833fcf6da63d1419babf4d62cf99429f755d5}{6ac833f} & \detokenize{05 May 2017 01:53 PM} & \detokenize{Tanner Cecchetti} & \detokenize{Merge branch 'master' of https://github.com/stribyw/CS444_producer-consumer}\\\hline
\href{https://github.com/stribyw/CS444_producer-consumer/commit/e9a97f3209bcd3109176ea21f62438c2720565bc}{e9a97f3} & \detokenize{05 May 2017 01:53 PM} & \detokenize{Tanner Cecchetti} & \detokenize{Add Makefile for synchro2}\\\hline
\href{https://github.com/stribyw/CS444_producer-consumer/commit/c87474154ac02391cdd3c49e891acd8d7d26e2a0}{c874741} & \detokenize{04 May 2017 08:52 PM} & \detokenize{William StribyJr} & \detokenize{changed filename}\\\hline
\href{https://github.com/stribyw/CS444_producer-consumer/commit/29a97b54f6399bedad2d7a5c53077b1be27b5fd1}{29a97b5} & \detokenize{04 May 2017 08:44 PM} & \detokenize{William StribyJr} & \detokenize{changed test file to read and write}\\\hline
\href{https://github.com/stribyw/CS444_producer-consumer/commit/efb041013e36e29767370c88638df2c5c46e5c6b}{efb0410} & \detokenize{04 May 2017 08:32 PM} & \detokenize{William StribyJr} & \detokenize{update local Merge branch 'master' of https://github.com/stribyw/CS444_Group10-01}\\\hline
\href{https://github.com/stribyw/CS444_producer-consumer/commit/1f46fa6da433a70a42a1e7ea18abf14417ebc79a}{1f46fa6} & \detokenize{04 May 2017 08:32 PM} & \detokenize{William StribyJr} & \detokenize{added test file that shuffles to write}\\\hline
\href{https://github.com/stribyw/CS444_producer-consumer/commit/506cc3e5dc3c8ed95b4014a9b4846f822223e7c0}{506cc3e} & \detokenize{04 May 2017 08:24 PM} & \detokenize{Tanner Cecchetti} & \detokenize{Update plot script}\\\hline
\href{https://github.com/stribyw/CS444_producer-consumer/commit/c1960832972cfbd27e327a8d1c3a0b3db46d53eb}{c196083} & \detokenize{04 May 2017 07:53 PM} & \detokenize{William StribyJr} & \detokenize{update files to push Merge branch 'master' of https://github.com/stribyw/CS444_Group10-01}\\\hline
\href{https://github.com/stribyw/CS444_producer-consumer/commit/e9c39e3fa729f084e14f8ee205c73610629bf7e7}{e9c39e3} & \detokenize{04 May 2017 07:52 PM} & \detokenize{Shannon Ernst} & \detokenize{named the philosophers}\\\hline
\href{https://github.com/stribyw/CS444_producer-consumer/commit/301901512c3dd2f16c706d252e2a8a5df3123da0}{3019015} & \detokenize{04 May 2017 07:51 PM} & \detokenize{William StribyJr} & \detokenize{Merge branch 'master' of https://github.com/stribyw/CS444_Group10-01}\\\hline
\href{https://github.com/stribyw/CS444_producer-consumer/commit/a4783728d8882e59716f658cdf57d71f890bd750}{a478372} & \detokenize{04 May 2017 07:51 PM} & \detokenize{Tanner} & \detokenize{Move tools to other dir}\\\hline
\href{https://github.com/stribyw/CS444_producer-consumer/commit/bad45a37c0655f452c1917da7200993b5b212aa9}{bad45a3} & \detokenize{04 May 2017 07:50 PM} & \detokenize{William StribyJr} & \detokenize{python test file to test sstf scheduler}\\\hline
\href{https://github.com/stribyw/CS444_producer-consumer/commit/5e0dadd2c3e14745fd1a27885c3a67feb151c0c3}{5e0dadd} & \detokenize{04 May 2017 07:48 PM} & \detokenize{Tanner Cecchetti} & \detokenize{Merge branch 'master' of https://github.com/stribyw/CS444_producer-consumer}\\\hline
\href{https://github.com/stribyw/CS444_producer-consumer/commit/93ed0231966ccd1e40e96adf15f1ffd4734ccf66}{93ed023} & \detokenize{04 May 2017 07:48 PM} & \detokenize{Shannon Ernst} & \detokenize{updating report}\\\hline
\href{https://github.com/stribyw/CS444_producer-consumer/commit/24957aba9d0526ddf608c9036fa197cfd29b0774}{24957ab} & \detokenize{04 May 2017 07:47 PM} & \detokenize{Tanner Cecchetti} & \detokenize{Add plotting tools for HW2}\\\hline
\href{https://github.com/stribyw/CS444_producer-consumer/commit/8d6f30f96788d2d81e1575a1e823a7dd167beb69}{8d6f30f} & \detokenize{02 May 2017 05:03 PM} & \detokenize{Shannon Ernst} & \detokenize{tex file set up}\\\hline\end{tabular}

\begin{tabular}{ | p{2.5cm} | p{4cm} | p{3cm} | p{7cm} | }\hline\textbf{Commit} & \textbf{Date} & \textbf{Author} & \textbf{Description}\\\hline
\href{https://github.com/t-a-n-n-e-r/cs444-kernel/commit/a890cd96b7db2e56e7222c1b33bac17cfee8b167}{a890cd96b7d} & \detokenize{05 May 2017 02:52 PM} & \detokenize{Tanner} & \detokenize{Merge branch 'master' of https://github.com/t-a-n-n-e-r/cs444-kernel}\\\hline
\href{https://github.com/t-a-n-n-e-r/cs444-kernel/commit/3919b5c579b79f8c64c3a17e77319db4d3044f46}{3919b5c579b} & \detokenize{05 May 2017 02:52 PM} & \detokenize{Tanner} & \detokenize{Make our scheduler default}\\\hline
\href{https://github.com/t-a-n-n-e-r/cs444-kernel/commit/f715294506075aa8b097d46967ac0aa1164be665}{f7152945060} & \detokenize{05 May 2017 02:18 PM} & \detokenize{Tanner Cecchetti} & \detokenize{Add better description}\\\hline
\href{https://github.com/t-a-n-n-e-r/cs444-kernel/commit/ba551c30bc807ad7c3ae7f5127c65d4de4433a0e}{ba551c30bc8} & \detokenize{04 May 2017 09:29 PM} & \detokenize{Tanner} & \detokenize{Attempt to remove elevator sort}\\\hline
\href{https://github.com/t-a-n-n-e-r/cs444-kernel/commit/26d91537530fa857969eaf21f514dc7d3a2292c2}{26d91537530} & \detokenize{04 May 2017 07:06 PM} & \detokenize{Tanner} & \detokenize{Attempt to add log level}\\\hline
\href{https://github.com/t-a-n-n-e-r/cs444-kernel/commit/a9af80232f7adad8c2dea8f08e1be7c2fcc99e53}{a9af80232f7} & \detokenize{03 May 2017 11:10 PM} & \detokenize{Tanner} & \detokenize{Fix small logical error}\\\hline
\href{https://github.com/t-a-n-n-e-r/cs444-kernel/commit/078545fafb8d2a0ff167cb1e604bae4853fdf4a8}{078545fafb8} & \detokenize{03 May 2017 10:55 PM} & \detokenize{Tanner} & \detokenize{Add test print statement}\\\hline
\href{https://github.com/t-a-n-n-e-r/cs444-kernel/commit/e34a132f7d4fce356e1518e9f914dd0e21192587}{e34a132f7d4} & \detokenize{03 May 2017 08:51 PM} & \detokenize{Tanner} & \detokenize{Improve formatting so numbers align}\\\hline
\href{https://github.com/t-a-n-n-e-r/cs444-kernel/commit/ce03ba7bb007186bf5c550a5bb94ddbd55913b90}{ce03ba7bb00} & \detokenize{03 May 2017 08:39 PM} & \detokenize{Tanner} & \detokenize{Add more meaningful print statements}\\\hline
\href{https://github.com/t-a-n-n-e-r/cs444-kernel/commit/5a1ec241c05ebb149d1d514ea72dff359fed5c4a}{5a1ec241c05} & \detokenize{03 May 2017 08:24 PM} & \detokenize{Tanner} & \detokenize{Fix infinite loop}\\\hline
\href{https://github.com/t-a-n-n-e-r/cs444-kernel/commit/e0eda9ee7e557ca22ec1f27b91ed97bb69d4b070}{e0eda9ee7e5} & \detokenize{03 May 2017 08:17 PM} & \detokenize{Tanner} & \detokenize{Add print statements}\\\hline
\href{https://github.com/t-a-n-n-e-r/cs444-kernel/commit/401abb045156fd4f2c255dad0744b21385be2150}{401abb04515} & \detokenize{03 May 2017 08:11 PM} & \detokenize{Tanner} & \detokenize{Add empty list check}\\\hline
\href{https://github.com/t-a-n-n-e-r/cs444-kernel/commit/5b8e13a2eaf7efb3b4696da7a518028902d40541}{5b8e13a2eaf} & \detokenize{03 May 2017 08:05 PM} & \detokenize{Tanner} & \detokenize{Fix compilation error, hopefully}\\\hline
\href{https://github.com/t-a-n-n-e-r/cs444-kernel/commit/731830ad4255ed15d427480ffd64f9547f315008}{731830ad425} & \detokenize{03 May 2017 07:59 PM} & \detokenize{Tanner} & \detokenize{Change how sector is read}\\\hline
\href{https://github.com/t-a-n-n-e-r/cs444-kernel/commit/3b0b185ef269b93eefad11752bf93357109487ed}{3b0b185ef26} & \detokenize{03 May 2017 07:50 PM} & \detokenize{Tanner} & \detokenize{Fix issues with add request}\\\hline
\href{https://github.com/t-a-n-n-e-r/cs444-kernel/commit/53f60edb311c3e052db6668d30c1cef6f5e4a734}{53f60edb311} & \detokenize{03 May 2017 07:35 PM} & \detokenize{Tanner} & \detokenize{Attempt to implement add request function}\\\hline
\href{https://github.com/t-a-n-n-e-r/cs444-kernel/commit/92f0965da3e9990cdf409eca471c9aa20753a6ed}{92f0965da3e} & \detokenize{02 May 2017 04:59 PM} & \detokenize{Tanner} & \detokenize{add our scheduler to makefile}\\\hline\end{tabular}

\newpage
\end{document}
