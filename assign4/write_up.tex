\documentclass[letterpaper,10pt,serif,draftclsnofoot,onecolumn,compsoc,titlepage]{IEEEtran}

\usepackage{graphicx}
\usepackage{amssymb}
\usepackage{amsmath}
\usepackage{amsthm}
\usepackage{cite}
\usepackage{alltt}
\usepackage{float}
\usepackage{color}
\usepackage{url}

\usepackage{balance}
\usepackage[TABBOTCAP, tight]{subfigure}
\usepackage{enumitem}

\usepackage{geometry}
\geometry{margin=.75in}
\usepackage{hyperref}
%\usetikzlibrary{shapes, positioning, calc}
\usepackage{caption}
\usepackage{listings}
%\usepackage[utf8]{inputenc}
%pull in the necessary preamble matter for pygments output

%% The following metadata will show up in the PDF properties
% \hypersetup{
%   colorlinks = false,
%   urlcolor = black,
%   pdfauthor = {\name},
%   pdfkeywords = {cs311 ``operating systems'' files filesystem I/O},
%   pdftitle = {CS 311 Project 1: UNIX File I/O},
%   pdfsubject = {CS 311 Project 1},
%   pdfpagemode = UseNone
% }

\parindent = 0.0 in
\parskip = 0.1 in
\title{Homework 4 Write up}
\author{Shannon Ernst, Keith Stirby, Tanner Cecchetti\\ 9 June 2017 \\ CS 444 Spring 2017 \\ Group 10-01}
\begin{document}
\maketitle
\begin{abstract}
Comparing algorithms is fundamental to Computer Science. This project 
compares best fit to first fit in the slob aspect of the kernel. 
\end{abstract}
\newpage
\tableofcontents
\newpage
\section{Purpose and Learning}
The purpose of this assignment was to learn the trade offs between two algorithms: 
best fit and first fit. We also had to learn about custom system calls and 
testing vague pieces of information. 
\section{Design Choices}
For this assignment we needed to work in slob.c under the mm folder. Slob operates 
originally on a first-fit algorithm meaning that it checks the available memory space 
on a page and which ever comes up as being able to hold the new piece of memory gets 
the new piece of memory. We were instructed to change this to a best fit algorithm. 
The difference is that for every piece of memory to be allocated, all available memory 
will be searched for the fit which is most exact to that piece. This will take longer 
because it has to search everything but it will be more efficient with memory and reduce
 fragmentation. The first fit is fast but uses a lot of memory and has a high fragmentation 
 rate. To implement best fit we needed to change slob alloc. Slob alloc 
 will look through all of the files with available space and examine them for a best fit, 
 saving the one which is a best fit and then allocating to that location once we have looked 
 through everything. This was all relatively trivial to implement.
\section{Testing}
To test our code we adjusted the system call table to include new system calls so we 
could have control over the memory usage. After adding these system calls we wrote a 
script in c to boot both kernel images and call our custom system calls. This recorded 
the memory usage for both and outputted it in a printed fashion. The fact that both images 
booted and have different memory usages is what causes us to say our algorithm is correct. 
As expected, the best fit had less fragmentation where the first fit had more.
\section{Work Log}
We met Memorial Day to work on the concurrency and we outlined our work for the kernel project.
We met on the Wednesday to work on the kernel and completed most of it. Over the weekend Tanner 
completed the testing. We then met Monday to do the write up and submit the assignment. 
\section{Version Control Log}

\end{document}
