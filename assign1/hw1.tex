\documentclass[letterpaper,10pt,serif,draftclsnofoot,onecolumn,compsoc,titlepage]{article}%this needs to change to IEEEtran but there is a bug

\usepackage{graphicx}
\usepackage{amssymb}
\usepackage{amsmath}
\usepackage{amsthm}

\usepackage{alltt}
\usepackage{float}
\usepackage{color}
\usepackage{url}

\usepackage{balance}
\usepackage[TABBOTCAP, tight]{subfigure}
\usepackage{enumitem}
\usepackage{pstricks, pst-node}

\usepackage{geometry}
\geometry{margin=.75in}
\usepackage{hyperref}
\usepackage{tikz}
%\usetikzlibrary{shapes, positioning, calc}
\colorlet{lightgray}{gray!20}
\usepackage{caption}
\usepackage{listings}
%\usepackage[utf8]{inputenc}
%pull in the necessary preamble matter for pygments output
\usepackage{fancyvrb}
\usepackage{color}
\usepackage[latin1]{inputenc}


\makeatletter
\def\PY@reset{\let\PY@it=\relax \let\PY@bf=\relax%
    \let\PY@ul=\relax \let\PY@tc=\relax%
    \let\PY@bc=\relax \let\PY@ff=\relax}
\def\PY@tok#1{\csname PY@tok@#1\endcsname}
\def\PY@toks#1+{\ifx\relax#1\empty\else%
    \PY@tok{#1}\expandafter\PY@toks\fi}
\def\PY@do#1{\PY@bc{\PY@tc{\PY@ul{%
    \PY@it{\PY@bf{\PY@ff{#1}}}}}}}
\def\PY#1#2{\PY@reset\PY@toks#1+\relax+\PY@do{#2}}

\expandafter\def\csname PY@tok@gd\endcsname{\def\PY@tc##1{\textcolor[rgb]{0.63,0.00,0.00}{##1}}}
\expandafter\def\csname PY@tok@gu\endcsname{\let\PY@bf=\textbf\def\PY@tc##1{\textcolor[rgb]{0.50,0.00,0.50}{##1}}}
\expandafter\def\csname PY@tok@gt\endcsname{\def\PY@tc##1{\textcolor[rgb]{0.00,0.25,0.82}{##1}}}
\expandafter\def\csname PY@tok@gs\endcsname{\let\PY@bf=\textbf}
\expandafter\def\csname PY@tok@gr\endcsname{\def\PY@tc##1{\textcolor[rgb]{1.00,0.00,0.00}{##1}}}
\expandafter\def\csname PY@tok@cm\endcsname{\let\PY@it=\textit\def\PY@tc##1{\textcolor[rgb]{0.25,0.50,0.50}{##1}}}
\expandafter\def\csname PY@tok@vg\endcsname{\def\PY@tc##1{\textcolor[rgb]{0.10,0.09,0.49}{##1}}}
\expandafter\def\csname PY@tok@m\endcsname{\def\PY@tc##1{\textcolor[rgb]{0.40,0.40,0.40}{##1}}}
\expandafter\def\csname PY@tok@mh\endcsname{\def\PY@tc##1{\textcolor[rgb]{0.40,0.40,0.40}{##1}}}
\expandafter\def\csname PY@tok@go\endcsname{\def\PY@tc##1{\textcolor[rgb]{0.50,0.50,0.50}{##1}}}
\expandafter\def\csname PY@tok@ge\endcsname{\let\PY@it=\textit}
\expandafter\def\csname PY@tok@vc\endcsname{\def\PY@tc##1{\textcolor[rgb]{0.10,0.09,0.49}{##1}}}
\expandafter\def\csname PY@tok@il\endcsname{\def\PY@tc##1{\textcolor[rgb]{0.40,0.40,0.40}{##1}}}
\expandafter\def\csname PY@tok@cs\endcsname{\let\PY@it=\textit\def\PY@tc##1{\textcolor[rgb]{0.25,0.50,0.50}{##1}}}
\expandafter\def\csname PY@tok@cp\endcsname{\def\PY@tc##1{\textcolor[rgb]{0.74,0.48,0.00}{##1}}}
\expandafter\def\csname PY@tok@gi\endcsname{\def\PY@tc##1{\textcolor[rgb]{0.00,0.63,0.00}{##1}}}
\expandafter\def\csname PY@tok@gh\endcsname{\let\PY@bf=\textbf\def\PY@tc##1{\textcolor[rgb]{0.00,0.00,0.50}{##1}}}
\expandafter\def\csname PY@tok@ni\endcsname{\let\PY@bf=\textbf\def\PY@tc##1{\textcolor[rgb]{0.60,0.60,0.60}{##1}}}
\expandafter\def\csname PY@tok@nl\endcsname{\def\PY@tc##1{\textcolor[rgb]{0.63,0.63,0.00}{##1}}}
\expandafter\def\csname PY@tok@nn\endcsname{\let\PY@bf=\textbf\def\PY@tc##1{\textcolor[rgb]{0.00,0.00,1.00}{##1}}}
\expandafter\def\csname PY@tok@no\endcsname{\def\PY@tc##1{\textcolor[rgb]{0.53,0.00,0.00}{##1}}}
\expandafter\def\csname PY@tok@na\endcsname{\def\PY@tc##1{\textcolor[rgb]{0.49,0.56,0.16}{##1}}}
\expandafter\def\csname PY@tok@nb\endcsname{\def\PY@tc##1{\textcolor[rgb]{0.00,0.50,0.00}{##1}}}
\expandafter\def\csname PY@tok@nc\endcsname{\let\PY@bf=\textbf\def\PY@tc##1{\textcolor[rgb]{0.00,0.00,1.00}{##1}}}
\expandafter\def\csname PY@tok@nd\endcsname{\def\PY@tc##1{\textcolor[rgb]{0.67,0.13,1.00}{##1}}}
\expandafter\def\csname PY@tok@ne\endcsname{\let\PY@bf=\textbf\def\PY@tc##1{\textcolor[rgb]{0.82,0.25,0.23}{##1}}}
\expandafter\def\csname PY@tok@nf\endcsname{\def\PY@tc##1{\textcolor[rgb]{0.00,0.00,1.00}{##1}}}
\expandafter\def\csname PY@tok@si\endcsname{\let\PY@bf=\textbf\def\PY@tc##1{\textcolor[rgb]{0.73,0.40,0.53}{##1}}}
\expandafter\def\csname PY@tok@s2\endcsname{\def\PY@tc##1{\textcolor[rgb]{0.73,0.13,0.13}{##1}}}
\expandafter\def\csname PY@tok@vi\endcsname{\def\PY@tc##1{\textcolor[rgb]{0.10,0.09,0.49}{##1}}}
\expandafter\def\csname PY@tok@nt\endcsname{\let\PY@bf=\textbf\def\PY@tc##1{\textcolor[rgb]{0.00,0.50,0.00}{##1}}}
\expandafter\def\csname PY@tok@nv\endcsname{\def\PY@tc##1{\textcolor[rgb]{0.10,0.09,0.49}{##1}}}
\expandafter\def\csname PY@tok@s1\endcsname{\def\PY@tc##1{\textcolor[rgb]{0.73,0.13,0.13}{##1}}}
\expandafter\def\csname PY@tok@sh\endcsname{\def\PY@tc##1{\textcolor[rgb]{0.73,0.13,0.13}{##1}}}
\expandafter\def\csname PY@tok@sc\endcsname{\def\PY@tc##1{\textcolor[rgb]{0.73,0.13,0.13}{##1}}}
\expandafter\def\csname PY@tok@sx\endcsname{\def\PY@tc##1{\textcolor[rgb]{0.00,0.50,0.00}{##1}}}
\expandafter\def\csname PY@tok@bp\endcsname{\def\PY@tc##1{\textcolor[rgb]{0.00,0.50,0.00}{##1}}}
\expandafter\def\csname PY@tok@c1\endcsname{\let\PY@it=\textit\def\PY@tc##1{\textcolor[rgb]{0.25,0.50,0.50}{##1}}}
\expandafter\def\csname PY@tok@kc\endcsname{\let\PY@bf=\textbf\def\PY@tc##1{\textcolor[rgb]{0.00,0.50,0.00}{##1}}}
\expandafter\def\csname PY@tok@c\endcsname{\let\PY@it=\textit\def\PY@tc##1{\textcolor[rgb]{0.25,0.50,0.50}{##1}}}
\expandafter\def\csname PY@tok@mf\endcsname{\def\PY@tc##1{\textcolor[rgb]{0.40,0.40,0.40}{##1}}}
\expandafter\def\csname PY@tok@err\endcsname{\def\PY@bc##1{\setlength{\fboxsep}{0pt}\fcolorbox[rgb]{1.00,0.00,0.00}{1,1,1}{\strut ##1}}}
\expandafter\def\csname PY@tok@kd\endcsname{\let\PY@bf=\textbf\def\PY@tc##1{\textcolor[rgb]{0.00,0.50,0.00}{##1}}}
\expandafter\def\csname PY@tok@ss\endcsname{\def\PY@tc##1{\textcolor[rgb]{0.10,0.09,0.49}{##1}}}
\expandafter\def\csname PY@tok@sr\endcsname{\def\PY@tc##1{\textcolor[rgb]{0.73,0.40,0.53}{##1}}}
\expandafter\def\csname PY@tok@mo\endcsname{\def\PY@tc##1{\textcolor[rgb]{0.40,0.40,0.40}{##1}}}
\expandafter\def\csname PY@tok@kn\endcsname{\let\PY@bf=\textbf\def\PY@tc##1{\textcolor[rgb]{0.00,0.50,0.00}{##1}}}
\expandafter\def\csname PY@tok@mi\endcsname{\def\PY@tc##1{\textcolor[rgb]{0.40,0.40,0.40}{##1}}}
\expandafter\def\csname PY@tok@gp\endcsname{\let\PY@bf=\textbf\def\PY@tc##1{\textcolor[rgb]{0.00,0.00,0.50}{##1}}}
\expandafter\def\csname PY@tok@o\endcsname{\def\PY@tc##1{\textcolor[rgb]{0.40,0.40,0.40}{##1}}}
\expandafter\def\csname PY@tok@kr\endcsname{\let\PY@bf=\textbf\def\PY@tc##1{\textcolor[rgb]{0.00,0.50,0.00}{##1}}}
\expandafter\def\csname PY@tok@s\endcsname{\def\PY@tc##1{\textcolor[rgb]{0.73,0.13,0.13}{##1}}}
\expandafter\def\csname PY@tok@kp\endcsname{\def\PY@tc##1{\textcolor[rgb]{0.00,0.50,0.00}{##1}}}
\expandafter\def\csname PY@tok@w\endcsname{\def\PY@tc##1{\textcolor[rgb]{0.73,0.73,0.73}{##1}}}
\expandafter\def\csname PY@tok@kt\endcsname{\def\PY@tc##1{\textcolor[rgb]{0.69,0.00,0.25}{##1}}}
\expandafter\def\csname PY@tok@ow\endcsname{\let\PY@bf=\textbf\def\PY@tc##1{\textcolor[rgb]{0.67,0.13,1.00}{##1}}}
\expandafter\def\csname PY@tok@sb\endcsname{\def\PY@tc##1{\textcolor[rgb]{0.73,0.13,0.13}{##1}}}
\expandafter\def\csname PY@tok@k\endcsname{\let\PY@bf=\textbf\def\PY@tc##1{\textcolor[rgb]{0.00,0.50,0.00}{##1}}}
\expandafter\def\csname PY@tok@se\endcsname{\let\PY@bf=\textbf\def\PY@tc##1{\textcolor[rgb]{0.73,0.40,0.13}{##1}}}
\expandafter\def\csname PY@tok@sd\endcsname{\let\PY@it=\textit\def\PY@tc##1{\textcolor[rgb]{0.73,0.13,0.13}{##1}}}

\def\PYZbs{\char`\\}
\def\PYZus{\char`\_}
\def\PYZob{\char`\{}
\def\PYZcb{\char`\}}
\def\PYZca{\char`\^}
\def\PYZam{\char`\&}
\def\PYZlt{\char`\<}
\def\PYZgt{\char`\>}
\def\PYZsh{\char`\#}
\def\PYZpc{\char`\%}
\def\PYZdl{\char`\$}
\def\PYZti{\char`\~}
% for compatibility with earlier versions
\def\PYZat{@}
\def\PYZlb{[}
\def\PYZrb{]}
\makeatother


%% The following metadata will show up in the PDF properties
% \hypersetup{
%   colorlinks = false,
%   urlcolor = black,
%   pdfauthor = {\name},
%   pdfkeywords = {cs311 ``operating systems'' files filesystem I/O},
%   pdftitle = {CS 311 Project 1: UNIX File I/O},
%   pdfsubject = {CS 311 Project 1},
%   pdfpagemode = UseNone
% }

\parindent = 0.0 in
\parskip = 0.1 in
\title{Homework 1 Write Up}
\author{Shannon Ernst, Keith Stirby, Tanner Cecchetti \\ CS 444 Spring 2017 \\ Group 10-01}
\begin{document}
\maketitle
\begin{abstract}
This document summarizes the work we completed on the first homework assignment.
It includes information such as: how we compiled and booted the Linux kernel,
a summary of QEMU command line parameters, and a summary of our concurrency assignment
solution. Lastly, it also contains a work log which itemizes the time we spent on
the assignment, and a version control log, which provides similar information.
\end{abstract}
\newpage
\section{Log of Commands}
\$ cd /scratch/spring2017 \\
\$ mkdir 10-01 \\
\$ chmod 777 10-01 \\
\$ cd 10-01 \\
\$ git clone git://git.yoctoproject.org/linux-yocto-3.14 \\
\$ cd linux-yocto-3.14 \\
\$ git checkout v3.14.26 \\
\$ source /scratch/opt/environment-setup-i586-poky-linux.csh \\
\$ cp /scratch/spring2017/files/config-3.14.26-yocto-qemu .config \\
\$ make menuconfig \\
pressed /, and typed LOCALVERSION, and pressed enter \\
hit 1, pressed enter, edited value to be -10-01-hw1 \\
\$ make -j4 all \\
\$ cd .. \\
\$ cp ./linux-yocto-3.14/arch/x86/boot/bzImage ./files/ \\
\$ qemu-system-i386 -gdb tcp::6501 -S -nographic -kernel ./files/bzImage -drive
file=core-image-lsb-sdk-qemux86.ext3,if=virtio -enable-kvm -net none -usb
localtime --no-reboot --append "root=/dev/vda rw console=ttyS0 debug" \\
\\ Open a new shell... \\
\$ source /scratch/opt/environment-setup-i586-poky-linux.csh \\
\$ \$GDB \\
\$ target remote :6501 \\
\$ continue \\
\\ Return to other shell... \\
kernel should have booted \\
login as root (no password) \\
\$ uname -a \\
\$ shutdown -h now
\section{qemu Flag Explanation}
-gdb tcp::6501 Flag waits for gdb connection on port 6501 \\
-S Flag does not start CPU at startup, you must type 'c' in the monitor \\
-nographic Flag disables graphical output so QEMU is a command line application \\
-kernel bzImage-qemux86.bin Flag specifies the image to boot, and can be either
a Linux kernel or in multiboot format \\
-drive file=core-image-lsb-sdk-qemux86.ext3,if=virtio Flag defines a new drive
with file core-image-lsb-sdk-qemux86.ext3 over the virtio interface \\
-enable-kvm Flag enables KVM full virtualization support \\
-net none Flag indicates that no network devices are configured \\
-usb Flag enables the USB driver \\
-localtime Flag is required to use the systems date and time \\
--no-reboot Flag exits instead of rebooting \\
--append "root=/dev/vda rw console=ttyS0 debug" Flag uses the command line as a
kernel command line \\
\section{Concurrency Write Up}
\subsection{Main Point of the Assignment}
The main point of the assignment was to introduce the idea of
concurrency and refresh on threads. The basics of the assignment was
to spawn a provided number of threads which would arbitrarily be assigned
 producer or consumer status. These threads would then operate on the same
 buffer space adding and removing jobs. This simple simulation allowed
 the student to become comfortable with pthreads.
\subsection{Design Decisions}
There were very few concrete requirements for this assignment. The few we
did have were:
\begin{itemize}
\item Producers create an object containing a random number and a random
wait time ranging from 2-9 seconds.
\item Producers add their object to a buffer so long as there are fewer
than 32 items in the buffer after a random wait time of 3-7 seconds.
\item Consumers remove jobs from the buffer so long as there are more than
zero in the buffer
\item Consumers read the wait time on a given item and wait for that amount
of time before printing the item's random value.
\item Randomization technique is determined by the system and will either
use rdrand or the Mersenne Twister.
\end{itemize}
Beyond these requirements, we were left to design the rest. We decided to
begin with the synchronization technique. This assignment needs to have
all of the threads running at the same time on the same buffer. We decided
to use mutexes to control the locking and unlocking of the buffer. A
producer will take a lock on the buffer to check if it is full. If it is
the lock is released immediately. If the buffer is not full, the producer
maintains the lock and will create an object. When an object is created
the program checks if it can use rdrand and then makes a selection on
which randomization technique to use. The item is then added to the current
counter position in the buffer and then the counter is incremented. The
buffer is an array which acts as a stack: the first item will be the last
item removed. We made this decision for simplicity's sake and recognize
that a better scheduler would be a first in first out structure which
would have just required a rotation of adding to the array, keeping two
index counters or pointers to keep track of where to consume from and
where to produce to. Similar to producers, consumers take a lock on the
buffer and check if it is empty. If it is empty, the consumer releases the
lock. If there is an item in the buffer the consumer removes it,
decrementing the counter, and reads the wait time. The consumer calls the
sleep command for that amount of time. When the consumer wakes up it prints
the value. The program continues until a kill command is received from the
keyboard. The number of threads is received from the command line.
\subsection{Testing}
We did incremental testing as we developed. The trickiest thing to test was
the randomization techniques. We began by testing the rdrand locally on a
new MAC laptop which successfully invoked the randomization. Having
confirmed that our program would run the rdrand function, we moved onto
the os-class server and implemented the Mersenne Twister. We tested with a
variety of threads on the os-class maxing out at 4, printing out traces of
the functions being invoked.
\subsection{Take Aways}
Coming into this assignment, there were some reservations about implementing
 pthreads. This assignment has shown us that pthreads is not as difficult as
 we feared. We also learned how to correctly lock and unlock implicitly
 shared data.
\section{Version Control Log}
\begin{tabular}{ | p{1.5cm} | p{4cm} | p{3cm} | p{8cm} | }\hline\textbf{Commit} & \textbf{Date} & \textbf{Author} & \textbf{Description}\\\hline
\href{https://github.com/stribyw/CS444_producer-consumer/commit/f3e1bc1fe86626ff5d4bba8bee0e04765b89d5c5}{f3e1bc1} & \detokenize{21 April 2017 05:40 PM} & \detokenize{Tanner} & \detokenize{Add basic abstract, rmeove duplicate date from title page}\\\hline
\href{https://github.com/stribyw/CS444_producer-consumer/commit/361dbd3a8c246de4513a17789a3614277259ae7c}{361dbd3} & \detokenize{21 April 2017 10:17 AM} & \detokenize{Tanner} & \detokenize{Reformat and tweak command log + flag explanations}\\\hline
\href{https://github.com/stribyw/CS444_producer-consumer/commit/0922e9feed8b70dbf8954676c77ecae2bb40aec9}{0922e9f} & \detokenize{21 April 2017 08:49 AM} & \detokenize{William StribyJr} & \detokenize{Updated hw1.tex file with commands used and flag explanations.}\\\hline
\href{https://github.com/stribyw/CS444_producer-consumer/commit/3c1fc5ceaf5083dadc8bc2d271f1c9480d293b17}{3c1fc5c} & \detokenize{21 April 2017 01:12 AM} & \detokenize{Tanner} & \detokenize{Change changelog table formatting}\\\hline
\href{https://github.com/stribyw/CS444_producer-consumer/commit/1df6fa570dbbbb2379a6c2156fa4a3c993fe1755}{1df6fa5} & \detokenize{21 April 2017 01:07 AM} & \detokenize{Tanner} & \detokenize{Add VC log to hw1 tex, add script to generate changelog.tex (adapted from online)}\\\hline
\href{https://github.com/stribyw/CS444_producer-consumer/commit/5591c72052a5108435cbeb2023be80bf567d7334}{5591c72} & \detokenize{21 April 2017 12:29 AM} & \detokenize{Tanner} & \detokenize{Remove pdf file from repo, ignore others with gitignore}\\\hline
\href{https://github.com/stribyw/CS444_producer-consumer/commit/19bbce9b44d141691cd52341aefe8a0efec88b1c}{19bbce9} & \detokenize{21 April 2017 12:26 AM} & \detokenize{Tanner Cecchetti} & \detokenize{Generate hw1 pdf with os-class}\\\hline
\href{https://github.com/stribyw/CS444_producer-consumer/commit/b325bcf64b8a5c54753a8faeeb996c2960432e6c}{b325bcf} & \detokenize{21 April 2017 12:24 AM} & \detokenize{Tanner} & \detokenize{Rework gitignore files, update hw1 LaTex (convert work log to table)}\\\hline
\href{https://github.com/stribyw/CS444_producer-consumer/commit/b7bc3e6388c6170eaf2083e1e5d5eaa72c5d1845}{b7bc3e6} & \detokenize{20 April 2017 11:47 PM} & \detokenize{Tanner Cecchetti} & \detokenize{Add initial LaTeX files}\\\hline
\href{https://github.com/stribyw/CS444_producer-consumer/commit/8151525199fb61c036f8020ee741d223ecce4b3e}{8151525} & \detokenize{20 April 2017 07:49 PM} & \detokenize{William StribyJr} & \detokenize{Changed permissions on files for everyone to read, write, & execute. (git status showed all files being modified so that's why I included this.) Styled and commented producer_consumer.c program. Removed the old producer_consumer.c program.}\\\hline
\href{https://github.com/stribyw/CS444_producer-consumer/commit/2a0472867d82d3506479f0aa6459d242ba871897}{2a04728} & \detokenize{20 April 2017 03:49 PM} & \detokenize{William StribyJr} & \detokenize{merging to pull}\\\hline
\href{https://github.com/stribyw/CS444_producer-consumer/commit/023aaf87b48d76f421dec5a52cc1c0b878aa3779}{023aaf8} & \detokenize{20 April 2017 09:16 AM} & \detokenize{Tanner} & \detokenize{Tweak README}\\\hline
\href{https://github.com/stribyw/CS444_producer-consumer/commit/0bc74168cfa5982d8e48c791d8ec9894349aef11}{0bc7416} & \detokenize{20 April 2017 09:15 AM} & \detokenize{Tanner} & \detokenize{Add message}\\\hline
\href{https://github.com/stribyw/CS444_producer-consumer/commit/6e983eb15cccd6c204251a5eb9d8bf95c581c471}{6e983eb} & \detokenize{20 April 2017 09:14 AM} & \detokenize{Tanner} & \detokenize{Add global gitignore, add synchro2 code}\\\hline
\href{https://github.com/stribyw/CS444_producer-consumer/commit/7fd57bdc87dba5b1478c36eb7bd9c72dc52cd825}{7fd57bd} & \detokenize{20 April 2017 09:11 AM} & \detokenize{Tanner} & \detokenize{Tweak README}\\\hline
\href{https://github.com/stribyw/CS444_producer-consumer/commit/dbe73dc7b79e0f572d098598f98c0d6c8d06e990}{dbe73dc} & \detokenize{20 April 2017 09:10 AM} & \detokenize{Tanner} & \detokenize{Move gitignore to correct directory}\\\hline
\href{https://github.com/stribyw/CS444_producer-consumer/commit/f82acf58e719d648b9d1cda4931106eb224c9cfb}{f82acf5} & \detokenize{20 April 2017 09:09 AM} & \detokenize{Tanner} & \detokenize{Reorganize repo}\\\hline
\href{https://github.com/stribyw/CS444_producer-consumer/commit/c153445a8ed5501f566c4ec34f1d1aad22559614}{c153445} & \detokenize{20 April 2017 09:00 AM} & \detokenize{Tanner} & \detokenize{Convert to buffer to FIFO}\\\hline
\href{https://github.com/stribyw/CS444_producer-consumer/commit/00448153f676ce120cd81fa08174bdd9f4cdf663}{0044815} & \detokenize{20 April 2017 08:40 AM} & \detokenize{Tanner} & \detokenize{Split Mersenne Twister into separate file, misc cleanup, fix indentation}\\\hline
\href{https://github.com/stribyw/CS444_producer-consumer/commit/f9ada64dcff4eb753b568062544266393fe1cf87}{f9ada64} & \detokenize{19 April 2017 08:02 PM} & \detokenize{Tanner} & \detokenize{update}\\\hline
\href{https://github.com/stribyw/CS444_producer-consumer/commit/d742cc2029756efe29eab44ed074a59039766d05}{d742cc2} & \detokenize{19 April 2017 07:37 PM} & \detokenize{Tanner} & \detokenize{Add makefile and gitignore}\\\hline
\href{https://github.com/stribyw/CS444_producer-consumer/commit/7e68e15e9da7eb0d4eacdc2b09d7d4b1de662e62}{7e68e15} & \detokenize{19 April 2017 07:34 PM} & \detokenize{Tanner} & \detokenize{update, kind of works}\\\hline
\href{https://github.com/stribyw/CS444_producer-consumer/commit/71ce6070d3e76cd824d4a9990b52ed4249ffb21f}{71ce607} & \detokenize{19 April 2017 07:13 PM} & \detokenize{William StribyJr} & \detokenize{update file}\\\hline
\href{https://github.com/stribyw/CS444_producer-consumer/commit/e11c0477d0ac10d95ec6f4530f28218fe810b1af}{e11c047} & \detokenize{19 April 2017 07:11 PM} & \detokenize{stribyw} & \detokenize{Add files via upload}\\\hline
\href{https://github.com/stribyw/CS444_producer-consumer/commit/1675a50845dbad185e43977150cdf2dbb33c1486}{1675a50} & \detokenize{19 April 2017 07:10 PM} & \detokenize{stribyw} & \detokenize{Add files via upload}\\\hline
\href{https://github.com/stribyw/CS444_producer-consumer/commit/f324e014d2377d8afd9fa4037993046f16c188d8}{f324e01} & \detokenize{19 April 2017 06:50 PM} & \detokenize{stribyw} & \detokenize{Add files via upload}\\\hline
\href{https://github.com/stribyw/CS444_producer-consumer/commit/66447fa4fc95a44e84741e28e7fc8d0350bc3a07}{66447fa} & \detokenize{19 April 2017 06:49 PM} & \detokenize{stribyw} & \detokenize{Initial commit}\\\hline\end{tabular}

\section{Work Log}
\begin{tabular}{| p{4.5cm} | p{1.5cm} | p{10.5cm} |}
\hline
Date & Time & Work performed \\ \hline
Tuesday, 11 April 2017 & 10:00 am & We met our group for the first time in
 recitation, we attempted to set up the vm and kernel boot in recitation
  unsuccessfully \\ \hline
Wednesday, 12 April 2017 & 6:00 pm & We met to successfully boot the vm and
the kernel. We were able to get the vm to work but could not get the kernel
to boot. We were going to ask questions later about it.\\ \hline
Wednesday, 19 April 2017 & 6:00 pm & We met and coded the concurrency
assignment.\\ \hline
Thursday, 20 April 2017 & 10:00 am & Tanner continued to work on the
concurrency implementation. Switched buffer implementation from stack to
queue, as well as abstracted the external random number generator library.
Additionally, restructured the project Git repository. \\ \hline
Thursday, 20 April 2017 & 9:00 pm & Shannon created the latex file, the
make file and wrote the concurrency write up section. \\ \hline
\end{tabular}

\end{document}
